\section{Development Plan for EpidCRN Package Enhancement}

Based on comprehensive analysis of the package architecture and current capabilities, we propose a focused three-phase improvement strategy targeting performance, completeness, and usability.

\subsection{Phase 1: Performance Optimization (Core \& Boundary subpackages)}

\textbf{Objective:} Address computational bottlenecks in symbolic boundary analysis.

\textbf{Rationale:} Current implementation experiences timeouts (10s for Solve, 3s for elimination) and slow performance for high-dimensional systems, as documented in Section 5.11-5.12.

\textbf{Improvements:}
\begin{enumerate}
\item \textbf{Caching Strategy}: Implement memoization for repeated \texttt{minSiph} and \texttt{NGM} computations
\begin{itemize}
\item Cache siphon analysis results indexed by reaction network structure
\item Store NGM eigenvalues/eigenvectors for parametric families
\end{itemize}

\item \textbf{Symbolic Simplification Pipeline}: Add preprocessing layer in \texttt{bdFp}
\begin{itemize}
\item Early detection of rational vs. algebraic cases before calling Solve
\item Gr\"obner basis preprocessing for polynomial systems
\item Automatic variable elimination ordering optimization
\end{itemize}

\item \textbf{Numerical-Symbolic Hybrid}: For complex systems exceeding timeout
\begin{itemize}
\item Numerical homotopy continuation as fallback for \texttt{"froze"} cases
\item Symbolic-numeric validation workflow
\end{itemize}
\end{enumerate}

\textbf{Success Metrics:} Reduce average computation time by 40\%, eliminate 80\% of timeout occurrences.

\subsection{Phase 2: Complete Invasion Graph Framework (Siphons \& new InvasionGraph subpackage)}

\textbf{Objective:} Implement and validate the invasion graph theory outlined in Section 6.

\textbf{Rationale:} Section 6 presents theoretical foundation but lacks complete implementation (\texttt{findAdmissibleCommunities} and \texttt{computeInvasionRates} are referenced but not defined). This framework unifies multi-strain dynamics analysis.

\textbf{Improvements:}
\begin{enumerate}
\item \textbf{Core Implementation}:
\begin{itemize}
\item Implement \texttt{findAdmissibleCommunities[RHS, var, mSi]} using siphon decomposition
\item Implement \texttt{computeInvasionRates[RHS, var, community, bdfpT]} connecting to boundary fixed points
\item Complete \texttt{rahmanInvasionGraph} with automatic susceptible level extraction
\end{itemize}

\item \textbf{Visualization Integration}:
\begin{itemize}
\item Create \texttt{plotInvasionGraph[communities, edges]} in Visualization subpackage
\item Add R0-annotated directed graphs showing invasion thresholds
\end{itemize}

\item \textbf{Validation Suite}:
\begin{itemize}
\item Test against classical models (2-strain SIS, competing strains, vaccine models)
\item Cross-validate with Lotka-Volterra equivalence theorem (Section 6.3)
\end{itemize}
\end{enumerate}

\textbf{Success Metrics:} Complete API with 5+ validated examples, integrate with existing \texttt{bd2} workflow.

\subsection{Phase 3: Enhanced User Experience (Utils \& Documentation)}

\textbf{Objective:} Improve package accessibility and error diagnostics.

\textbf{Rationale:} Critical usage notes (Section 8) suggest common pitfalls; timeout handling returns cryptic \texttt{"froze"} messages.

\textbf{Improvements:}
\begin{enumerate}
\item \textbf{Input Validation Layer}:
\begin{itemize}
\item Add \texttt{validateNetwork[RN, rts]} checking reaction-rate consistency
\item Validate siphon structure before expensive computations
\item Type checking for string vs. symbol usage (addresses critical note in Section 8)
\end{itemize}

\item \textbf{Diagnostic Tools}:
\begin{itemize}
\item \texttt{estimateComplexity[RN]} predicting computation time before analysis
\item \texttt{explainTimeout[bdfpT]} suggesting model simplifications when timeouts occur
\item Progress indicators for long-running \texttt{scan} and \texttt{scanPar} operations
\end{itemize}

\item \textbf{Interactive Examples}:
\begin{itemize}
\item Create \texttt{Examples/} directory with notebook tutorials
\item Document complete workflows for common model types (SIS, SIR, SEIR variants)
\item Add quickstart guide referencing Section 9 workflow
\end{itemize}
\end{enumerate}

\textbf{Success Metrics:} Reduce user-reported errors by 50\%, achieve 90\% successful first-run rate for documented examples.

\subsection{Implementation Priority}

We recommend sequential execution: Phase 1 $\rightarrow$ Phase 2 $\rightarrow$ Phase 3, as performance improvements enable broader testing in Phase 2, and both inform better user guidance in Phase 3. Estimated timeline: 6-8 weeks total (2.5 weeks per phase).
