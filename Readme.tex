\documentclass{article}
\usepackage[utf8]{inputenc}
\usepackage{amsmath,amsfonts,amssymb}
\usepackage{listings}
\usepackage{xcolor}
\usepackage{geometry}
\geometry{margin=1in}

\lstset{
    language=Mathematica,
    basicstyle=\small\ttfamily,
    keywordstyle=\color{blue},
    commentstyle=\color{gray},
    stringstyle=\color{red},
    breaklines=true,
    frame=single
}

\title{EpidCRN Package Documentation}
\author{Context File for AI Assistant}
\date{\today}

\begin{document}

\maketitle

\section{Package Overview}

\textbf{EpidCRN} is a Mathematica package for epidemiological models, which uses Chemical Reaction Network Theory (CRNT) methods. It  aims to analyze  models with unique disease-free boundary fixed point (DFE), possibly  multi-strain (ie. with other boundary fixed points besides the DFE), and with unique positive fixed point (endemic).

\section{Package Structure}

\subsection{Modular Organization}
The package is split into subpackages to reduce monolithic complexity:

\begin{itemize}
\item \texttt{EpidCRN.wl} - Main loader package
\item \texttt{Core.wl} - Basic network analysis (\texttt{EpidCRN`Core`})
\item \texttt{CRNT.wl} - Chemical reaction network theory (\texttt{EpidCRN`CRNT`})
\item \texttt{Boundary.wl} - NGM and boundary analysis (\texttt{EpidCRN`Boundary`})
\item \texttt{Bifurcation.wl} - Hopf bifurcations and parameter scanning (\texttt{EpidCRN`Bifurcation`})
\item \texttt{Siphons.wl} - Siphon and persistence analysis (\texttt{EpidCRN`Siphons`})
\item \texttt{Utils.wl} - Utility functions (\texttt{EpidCRN`Utils`})
\end{itemize}

\subsection{Dependency Chain}
\texttt{Core} $\rightarrow$ \texttt{CRNT} $\rightarrow$ \texttt{Boundary} $\rightarrow$ \texttt{Bifurcation} $\rightarrow$ \texttt{Siphons}

\section{Key Functions by Category}

\subsection{Core Functions (EpidCRN`Core`)}

\begin{itemize}
\item \texttt{extMat[reactions]} - Master function extracting \{species, $\alpha$, $\beta$, $\gamma$, $R_v$, RHS, deficiency\}
\item \texttt{compToAsso[side]} - Parses reaction side to association of species $\rightarrow$ coefficients
\item \texttt{extSpe[reactions]} - Extracts species list
\item \texttt{asoRea[RN]} - Converts to association format with "Substrates"/"Products" keys
\end{itemize}

\subsection{CRNT Functions (EpidCRN`CRNT`)}

\begin{itemize}
\item \texttt{getComE[RN]} - Extracts \{complexes, edges\} from reaction network
\item \texttt{IaFHJ[vertices, edges]} - Incidence matrix analysis, returns \{matrix, tableForm\}
\item \texttt{IkFHJ[vertices, edges, rates]} - Ik matrix (n\_reactions $\times$ n\_complexes)
\item \texttt{lapK[RN, rates]} - Laplacian matrix computation
\item \texttt{SpeComInc[complexes, species]} - Species-complex incidence matrix
\end{itemize}

\subsection{Boundary Analysis (EpidCRN`Boundary`)}

\begin{itemize}
\item \texttt{NGM[mod, inf]} - Next Generation Matrix analysis
\item \texttt{bd1[RN, rts]} - Single strain boundary analysis (DFE + 1 endemic)
\item \texttt{bd2[RN, rts]} - Two strain analysis (DFE + 2 boundary + 1 coexistence)
\item \texttt{DFE[mod, inf]} - Disease-Free Equilibrium computation
\item \texttt{mRts[RN, ks]} - Mass action rates with parameter names
\end{itemize}

\subsection{Bifurcation Analysis (EpidCRN`Bifurcation`)}

\begin{itemize}
\item \texttt{fpHopf[RHS, var, par, p0val]} - Fixed point finder with Hopf analysis, returns angle = ArcTan[Re/Im]*180/Pi
\item \texttt{simpleOptHopf[RHS, var, par, coP, optInd, numTries]} - Simple random search optimization for Hopf bifurcations
\item \texttt{optHopf[RHS, var, par, coP, optInd, timeLimit, method, accGoal, precGoal, maxIter]} - Sophisticated NMaximize-based Hopf optimization
\item \texttt{cont[RHS, var, par, p0val, stepSize, plotInd, bifInd, analyticalPlot]} - Continuation analysis with overlay capability
\item \texttt{hopfD[curve]} - Hopf bifurcation detection from continuation curve data
\item \texttt{scanPar[RHS, var, par, p0val, plotInd, gridRes, plot, ...]} - Comprehensive parameter space scanning with equilibrium classification
\item \texttt{pertIC[equilibrium, var, factor, minq, n]} - Generate perturbed initial conditions
\item \texttt{TS[RHS, var, par, p0val, tmax]} - Time series simulation using NDSolve
\item \texttt{intEq[RHS, var, par, p0val, att]} - Interactive equilibrium finding with perturbations
\end{itemize}

\section{Critical Workflows}

\subsection{Laplacian Matrix Construction}
For "invasion CRN" (network projected on classes that become 0 at DFE):

\begin{lstlisting}
{complexes, edges} = getComE[RN];
incidenceMatrix = IaFHJ[complexes, edges][[1]];
ikMatrix = IkFHJ[complexes, edges, rates];
laplacian = ikMatrix; (* or incidenceMatrix . ikMatrix *)
\end{lstlisting}

\subsection{Boundary Analysis Pipeline}
\begin{lstlisting}
(* Single strain *)
result1 = bd1[RN, rates];
{RHS, var, par, cp, mSi, Jx, Jy, E0, ngm, R0A, EA, E1} = result1;

(* Two strain *)
result2 = bd2[RN, rates];
{RHS, var, par, cp, mSi, Jx, Jy, E0, ngm, R0A, EA, E1t, E2t} = result2;
\end{lstlisting}

\subsection{Hopf Bifurcation Detection}
For detecting Hopf bifurcations through parameter optimization:

\begin{lstlisting}
(* Simple approach - random search *)
{bestAngle, bestValues, finalP0Val} =
  simpleOptHopf[RHS, var, par, coP, {3,4}, 50];

(* Sophisticated approach - NMaximize *)
{bestAngle, bestValues, finalP0Val} =
  optHopf[RHS, var, par, coP, {3,4}, 120, "NelderMead", 4, 4, 500];

(* Angle interpretation: negative = stable focus, positive = Hopf *)
\end{lstlisting}

\subsection{Continuation Analysis}
For tracking equilibria along parameter paths:

\begin{lstlisting}
(* Basic continuation along parameter 4 *)
curve = cont[RHS, var, par, p0val, 0.01, {3,4}, 4, None];

(* With analytical plot overlay *)
curve = cont[RHS, var, par, p0val, 0.01, {3,4}, 4, analyticalPlot];

(* Detect Hopf points from continuation curve *)
hopfPoints = hopfD[curve];
\end{lstlisting}

\subsection{Parameter Space Scanning}
For comprehensive equilibrium classification:

\begin{lstlisting}
(* Grid mode scanning *)
{plot, errors, results} = scanPar[RHS, var, par, p0val, {1,2},
  30, Automatic, 0.01, 1/20, 0.5, 0.5, R01, R02, R21, R12];

(* Range mode scanning *)
{plot, errors, results} = scanPar[RHS, var, par, p0val, {1,2},
  Automatic, basePlot, 0.01, 0.05, 1.0, 1.0, R01, R02, R21, R12];
\end{lstlisting}

\section{Important Implementation Details}

\subsection{Function Compatibility Issues}
\begin{itemize}
\item \texttt{IaFHJ} and \texttt{IkFHJ} use \texttt{Table[gg[vert[[i]], edg[[j]]], ...]} NOT \texttt{Outer[gg, vert, edg]} due to Part specification errors
\item Functions expect exact expression matching using \texttt{===} operator
\item Species can be strings ("S1"), expressions ("I1" + "S1"), or symbols depending on context
\end{itemize}

\subsection{Data Format Standards}
\begin{itemize}
\item Reactions: \texttt{\{\{leftSide, rightSide\}, ...\}} or \texttt{\{leftSide -> rightSide, ...\}}
\item Species: List of strings \texttt{\{"S1", "I1", "I2"\}}
\item Rates: List parallel to reactions \texttt{\{k1, k2, k3\}}
\item Returns: Most functions return lists, not associations (user preference)
\end{itemize}

\section{Key Concepts}

\subsection{Invasion Species}
Species that become zero at Disease-Free Equilibrium, typically infection classes. Found using \texttt{minSiph[species, asoRea[RN]]}.

\subsection{Boundary Analysis Types}
\begin{itemize}
\item \texttt{bd1}: Single strain - expects DFE and one endemic equilibrium
\item \texttt{bd2}: Two strain - expects DFE, two boundary points, one coexistence equilibrium
\end{itemize}

\subsection{Matrix Conventions}
\begin{itemize}
\item $\alpha$: reactant stoichiometric matrix (species $\times$ reactions)
\item $\beta$: product stoichiometric matrix
\item $\gamma = \beta - \alpha$: net stoichiometric matrix
\item Laplacian: follows CRNT conventions, NOT standard graph theory (row sums = 0)
\end{itemize}

\section{Removed/Deprecated Functions}

\begin{itemize}
\item \texttt{NGMs} - Removed (noted as treating "denominators and exponents incorrectly")
\item \texttt{bdAnalG}, \texttt{bdAnalG0} - Removed (unclear differences from bd1/bd2)
\item Multiple near-duplicate functions consolidated into master functions
\end{itemize}

\section{File Structure}

\begin{verbatim}
EpidCRN/
    EpidCRN.wl       (main package loader)
    Core.wl          (EpidCRN`Core` context)
    CRNT.wl          (EpidCRN`CRNT` context)
    Boundary.wl      (EpidCRN`Boundary` context)
    Bifurcation.wl   (EpidCRN`Bifurcation` context)
    Siphons.wl       (EpidCRN`Siphons` context)
    Utils.wl         (EpidCRN`Utils` context)
\end{verbatim}

\textbf{Critical}: File names must match subcontext names exactly for \texttt{Get[]} to work automatically.

\section{Usage Patterns}

\subsection{Loading}
\begin{lstlisting}
(* Load full package *)
Get["EpidCRN`"];

(* Load individual subpackage for testing *)
Get["EpidCRN`Core`"];
\end{lstlisting}

\subsection{Typical Analysis Sequence}
\begin{lstlisting}
(* 1. Extract network structure *)
{species, alpha, beta, gamma, Rv, RHS, deficiency} = extMat[RN];

(* 2. Boundary analysis *)
boundaryResults = bd2[RN, rates];

(* 3. Parameter space scanning *)
{plot, errors, results} = scanPar[RHS, variables, parameters, p0val,
  {1,2}, 30, Automatic];

(* 4. Hopf bifurcation search *)
{bestAngle, bestValues, finalP0Val} =
  optHopf[RHS, variables, parameters, coP, {3,4}, 120];

(* 5. Continuation analysis *)
curve = cont[RHS, variables, parameters, finalP0Val, 0.01, {3,4}, 4];
hopfPoints = hopfD[curve];

(* 6. Laplacian for invasion dynamics *)
{laplacian, complexes, edges} = lapK[RN, rates];

(* 7. NGM analysis *)
mod = {RHS, variables, parameters};
ngmResults = NGM[mod, infectionIndices];
\end{lstlisting}

\section{Notes for Future Development}

\begin{itemize}
\item User strongly prefers list returns over association returns
\item Focus on functions that work symbolically and numerically
\item Avoid vague descriptions like "complete analysis" - specify what differs between functions
\item Package grew organically with substantial duplication - consolidation ongoing
\item Core functions like \texttt{extMat} are dependency roots for most other functions
\item \texttt{Bifurcation} subpackage focuses on dynamical systems analysis - works numerically with NDSolve
\item Hopf detection uses angle criterion: negative = stable focus, positive = Hopf bifurcation
\item Parameter scanning supports both grid mode (fixed resolution) and range mode (adaptive stepping)
\item Continuation analysis can overlay results on analytical plots when bifurcation parameter is in plot indices
\end{itemize}

\end{document}